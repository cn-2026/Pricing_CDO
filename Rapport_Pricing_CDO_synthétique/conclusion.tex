\section{Conclusion}

Dans ce rapport, nous avons mis en place un cadre de modélisation pour la tarification des tranches de CDO synthétiques.
Après avoir rappelé le cadre risque-neutre et les éléments de base du marché du crédit (CDS, jambe premium et jambe protection),
nous avons présenté des modèles de dépendance des défauts via les copules (gaussienne et de Student) ainsi qu’un cadre à intensité.
Nous avons ensuite décrit une procédure numérique de calibration et appliqué ces méthodes aux tranches du CDX NA IG Series 19
(maturité 5 ans) sur dix dates de marché issues de \cite{OhkrinXu}.

Les résultats numériques confirment une limite classique des approches à un facteur sous hypothèse de portefeuille homogène :
une calibration globale par un seul paramètre de corrélation ne permet pas d’ajuster correctement toutes les tranches simultanément.
La copule gaussienne reproduit la tranche equity mais présente des écarts importants sur les tranches mezzanine et senior, et
l’apparition d’un smile de corrélation implicite indique que la dépendance demandée par le marché varie selon le niveau de subordination,
phénomène bien documenté dans la littérature sur les tranches d’indices de crédit \cite{OhkrinXu}. Le passage à la copule de Student
apporte davantage de flexibilité via le paramètre de degrés de liberté, mais dans notre paramétrisation simple il subsiste des écarts
significatifs sur les tranches mezzanine et senior. Cela suggère que la structure de dépendance est plus complexe que ce que capture un
modèle homogène à un seul facteur.

Ces constats motivent la poursuite du projet. Dans une deuxième partie, nous chercherons à améliorer l’ajustement des tranches mezzanine
et senior en enrichissant la structure de dépendance. D’une part, nous testerons des combinaisons de copules, comme proposé dans
\cite{OhkrinXu}, afin d’accroître la flexibilité du modèle et de mieux représenter les comportements de queue. D’autre part, nous
étudierons un modèle à intensité, notamment avec une composante systémique commune, dans l’esprit des modèles de CDO basés sur intensités
et facteurs de marché \cite{DuffieGarleanu}. L’objectif sera de comparer ces extensions sur les mêmes données, en termes de qualité de
calibration, de stabilité et de coût numérique.
