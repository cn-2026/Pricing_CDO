\section{Méthodes numériques}

\subsection{Simulation des taux d'intérêts}

\paragraph{Principe général de la simulation et différence entre les modèles.}

\vspace{0.5cm}

Dans les deux cas, les formules utilisées pour simuler $r_{t+\Delta t}$ correspondent à l’échantillonnage de la loi de transition exacte du taux court, et non à un schéma de discrétisation de type Euler.

Dans le modèle de Vasicek, l’équation étant linéaire, le processus est gaussien et la loi conditionnelle de $r_{t+\Delta t}$ sachant $r_t$ est normale, ce qui conduit à une formule de simulation explicite.  
Dans le modèle CIR, le coefficient de diffusion dépend de $\sqrt{r_t}$, l’équation n’est plus linéaire et la loi conditionnelle de transition est une loi du $\chi^2$ non centrale.

La différence entre les expressions de simulation provient donc directement de la structure des équations différentielles stochastiques des deux modèles.

\subsubsection{Simulation du modèle de Vasicek}

Le modèle de Vasicek est linéaire et admet une solution explicite sous forme d'intégrale stochastique :

\begin{equation}
    r_{t+\Delta t} = r_t e^{-a \Delta t} + b (1 - e^{-a \Delta t}) + \sigma \sqrt{\frac{1 - e^{-2 a \Delta t}}{2a}} \, Z_t,
\end{equation}

où $Z_t \sim \mathcal{N}(0,1)$ est une variable aléatoire normale standard indépendante.

\textbf{Méthode de simulation :}
\begin{enumerate}
    \item Choisir un pas de temps $\Delta t$ et un horizon final $T$.
    \item Initialiser $r_0$.
    \item Générer des variables normales $Z_0, Z_1, \dots$.
    \item Calculer itérativement $r_{t+\Delta t}$ à partir de $r_t$ en utilisant la formule ci-dessus.
\end{enumerate}


\medskip
\paragraph{Calcul du numéraire dans le modèle de Vasicek}

On rappelle que le numéraire associé au marché monétaire est défini par
\begin{equation}
    B_t=\exp\left(\int_0^t r_s\,ds\right).
\end{equation}

Dans le cadre du modèle de Vasicek, le processus $(r_t)_{t\geq 0}$ est un processus gaussien à trajectoires continues et vérifie la propriété d’intégrabilité
\[
\int_0^T |r_s|\,ds<\infty \quad \text{p.s.}
\]
comme établi dans la section précédente.

Par conséquent, pour toute trajectoire simulée $(r_{t_k})_{k=0,\ldots,N}$, une approximation naturelle du numéraire s’obtient par une discrétisation de l’intégrale :
\begin{equation}
    B_{t_{k+1}}
    =
    B_{t_k}\exp\!\left(r_{t_k}\Delta t\right),
    \qquad B_0=1,
\end{equation}
où $\Delta t=t_{k+1}-t_k$.

Cette formule découle directement de la définition intégrale de $B_t$ et du fait que, pour des trajectoires continues, la somme de Riemann
\[
\sum_{k=0}^{N-1} r_{t_k}\Delta t
\]
converge presque sûrement vers $\int_0^T r_s\,ds$ lorsque $\Delta t\to 0$.

Ainsi, la simulation exacte du taux court $r_t$ dans le modèle de Vasicek permet de calculer de manière cohérente le numéraire par simple intégration numérique le long des trajectoires simulées.



\subsubsection{Simulation du modèle CIR}

Le modèle CIR ne possède pas de solution intégrale simple, mais il est possible de simuler $r_t$ de manière exacte grâce à la relation avec la loi du $\chi^2$ non centrale.

\textbf{Méthode de simulation :}
\begin{enumerate}
    \item Choisir un pas de temps $\Delta t$ et un horizon $T$.
    \item Initialiser $r_0 > 0$.
    \item À chaque pas, générer $r_{t+\Delta t}$ selon :
    \[
        r_{t+\Delta t} = \frac{\sigma^2 (1 - e^{-a \Delta t})}{4a} \, \chi^2_{\nu}(\lambda),
    \]
    où $\nu = \frac{4ab}{\sigma^2}$ et $\lambda = \frac{4 a r_t e^{-a \Delta t}}{\sigma^2 (1 - e^{-a \Delta t})}$.
\end{enumerate}


\begin{itemize}
    \item La simulation des trajectoires du modèle de Vasicek est simple et rapide grâce à la solution analytique.
    \item Le modèle CIR garantit la positivité du taux, mais nécessite la génération de variables $\chi^2$ non centrales.
\end{itemize}

\medskip
\paragraph{Calcul du numéraire dans le modèle CIR}

Dans le modèle de Cox--Ingersoll--Ross, le numéraire est également défini par
\begin{equation}
    B_t=\exp\left(\int_0^t r_s\,ds\right).
\end{equation}

Sous les hypothèses du modèle, et en particulier sous la condition de Feller $2ab\geq \sigma^2$, le processus $r_t$ est strictement positif et à trajectoires continues.

On a donc, pour tout horizon fini $T$,
\[
\int_0^T r_s\,ds<\infty \quad \text{p.s.}
\]
ce qui garantit que le numéraire est bien défini et strictement positif.

Comme dans le cas de Vasicek, le calcul du numéraire repose sur une approximation de l’intégrale par une discrétisation temporelle. Pour une grille régulière $(t_k)$, on pose
\begin{equation}
    B_{t_{k+1}}
    =
    B_{t_k}\exp\!\left(r_{t_k}\Delta t\right),
    \qquad B_0=1.
\end{equation}

Cette approximation est justifiée par la continuité presque sûre de $r_t$, qui assure la convergence de la somme de Riemann vers l’intégrale stochastique déterministe
\[
\int_0^t r_s\,ds.
\]

Le fait que la simulation du processus CIR soit effectuée à l’aide de sa loi exacte à chaque pas de temps ne modifie pas ce calcul : l’intégrale du taux court ne possède pas de loi fermée simple et doit être évaluée numériquement le long des trajectoires simulées.




\subsection{Calibration des copules}
\subsection{Calibration de la copule gaussienne}

Nous calibrons une copule gaussienne à un facteur à partir des spreads observés sur les
10 dates du jeu de données. L’objectif est de choisir le paramètre de dépendance \(\rho\)
de manière à reproduire au mieux, pour chaque date, les spreads des différentes tranches.
Pour une date donnée, on cherche
\[
\rho^\star
= \arg\min_{\rho \in [\rho_{\min},\rho_{\max}]} \mathcal{L}(\rho),
\]
où \(\mathcal{L}\) mesure l’écart entre spreads de marché et spreads du modèle. Dans notre
implémentation, la fonction objectif est la RMSE sur les \(Q\) tranches :
\[
\mathcal{L}(\rho)
= \sqrt{\frac{1}{Q}\sum_{q=1}^Q \big(s^{\mathrm{mod}}_q(\rho)-s^{\mathrm{mkt}}_q\big)^2 }.
\]

Afin de simplifier le problème, nous adoptons l’approximation d’un portefeuille homogène :
toutes les paires de noms partagent la même corrélation \(\rho\), ce qui conduit à la matrice
équicorrélée
\[
\Sigma(\rho)=
\begin{pmatrix}
1 & \rho & \cdots & \rho \\
\rho & 1 & \cdots & \rho \\
\vdots & \vdots & \ddots & \vdots \\
\rho & \rho & \cdots & 1
\end{pmatrix}.
\]
On ne calibre alors qu’un seul paramètre.

Pour une valeur candidate \(\rho\), on génère des scénarios corrélés via
\(X \sim \mathcal{N}(0,\Sigma(\rho))\), puis on pose \(U_i=\Phi(X_i)\). Les temps de défaut
sont obtenus par inversion dans un modèle à intensité constante :
\[
\tau_i = \frac{-\ln(U_i)}{\lambda},
\qquad
\lambda \approx \frac{s_{\text{index}}}{1-R},
\]
où \(s_{\text{index}}\) est le spread de l’indice CDS et \(R\) le taux de recouvrement fixé.
Pour limiter la variabilité numérique entre deux évaluations de \(\mathcal{L}(\rho)\), on fixe
la graine aléatoire lors des simulations.

À partir des temps de défaut simulés, on calcule la perte cumulée du portefeuille aux dates
de paiement, puis la perte de tranche via les points d’attachement/détachement. Le spread
modèle \(s^{\mathrm{mod}}_q(\rho)\) est ensuite déterminé par la condition d’absence d’arbitrage
(égalité entre jambe de protection et jambe premium), en approchant les intégrales par une
somme discrète sur les dates de paiement.

L’optimisation est réalisée en deux étapes pour chaque date : (i) une recherche sur une grille
\(\rho \in [0.01,0.99]\) afin d’obtenir une initialisation \(\rho_{\text{init}}\), puis (ii) un
raffinement local par \texttt{scipy.optimize.minimize} avec l’algorithme \texttt{L-BFGS-B}
sous la contrainte \(\rho \in [0.01,0.99]\), afin d’obtenir \(\rho^\star\). Enfin, pour réduire
le coût numérique, les valeurs calibrées et les spreads modèle associés sont sauvegardés afin
d’éviter de relancer l’ensemble des calculs à chaque exécution.

\subsection{Calibration de la copule de Student (t-copula)}

Afin de mieux représenter les défauts simultanés observés en période de stress, nous
remplaçons la copule gaussienne par une copule de Student. Ce modèle ajoute un paramètre
\(\nu>0\) (degrés de liberté) qui contrôle l’épaisseur des queues : lorsque \(\nu\) est faible,
la dépendance extrême est plus marquée, et lorsque \(\nu \to \infty\), on retrouve le cas gaussien.

Comme précédemment, l’objectif est de reproduire les spreads de marché des \(Q\) tranches à
chaque date. La calibration consiste désormais à estimer \(\rho\) et \(\nu\) en minimisant
\[
(\rho^\star,\nu^\star)
= \arg\min_{(\rho,\nu)\in\mathcal{D}} \mathcal{L}(\rho,\nu),
\qquad
\mathcal{L}(\rho,\nu)
= \sqrt{\frac{1}{Q}\sum_{q=1}^Q\big(s_q^{\mathrm{mod}}(\rho,\nu)-s_q^{\mathrm{mkt}}\big)^2},
\]
où \(\mathcal{D}\) est un domaine borné (dans notre implémentation,
\(\rho\in[0.01,0.99]\) et \(\nu\in[2.01,80]\)).

On conserve l’hypothèse d’un portefeuille homogène via la matrice équicorrélée \(\Sigma(\rho)\).
Pour simuler la t-copule, on génère d’abord un vecteur gaussien équicorrélé \((Z_i)_{i=1}^n\)
à l’aide d’un facteur commun \(Y\sim\mathcal{N}(0,1)\) et de termes idiosyncratiques
\(\varepsilon_i\sim\mathcal{N}(0,1)\) indépendants :
\[
Z_i=\sqrt{\rho}\,Y+\sqrt{1-\rho}\,\varepsilon_i.
\]
On introduit ensuite une variable de mélange commune \(S\sim\chi^2(\nu)\), indépendante, et l’on pose
\[
T_i=\frac{Z_i}{\sqrt{S/\nu}},
\qquad
U_i=t_\nu(T_i),
\]
où \(t_\nu\) désigne la fonction de répartition de la Student univariée à \(\nu\) degrés de liberté.
Le vecteur \((U_i)\) suit alors une copule de Student de paramètres \((\rho,\nu)\). Les temps de défaut
sont obtenus en appliquant la même transformation marginale que dans le cas gaussien.

Comme la fonction \(\mathcal{L}(\rho,\nu)\) est estimée par Monte-Carlo, on fixe la graine aléatoire afin
de limiter la variabilité numérique entre deux évaluations. La recherche des paramètres est réalisée en
deux étapes : une exploration sur grille pour initialiser \((\rho_{\mathrm{init}},\nu_{\mathrm{init}})\),
puis un raffinement local via \texttt{scipy.optimize.minimize} avec l’algorithme \texttt{L-BFGS-B} sous
contraintes, conduisant à \((\rho^\star,\nu^\star)\).


\subsubsection{Modèle de Vasicek}

Dans le cas d'un portefolio homogène, grand et suffisament diversifié, il est possible
de faire l'approximation de Vasicek ou LHP (\emph{Large Homogeneous Portfolio}) qui permet de calculer les pertes espérées et donc
les spreads pour chaque tranche en fonction de la corrélation implicite. Pour cela on considère un portefeuille de crédit homogène de taille $n$.
La dépendance des défauts est décrite par un modèle gaussien à un facteur :
\begin{equation}
    X_i = \sqrt{\rho}\,Z + \sqrt{1-\rho}\,\varepsilon_i,
\end{equation}
où $Z,\varepsilon_i \sim \mathcal N(0,1)$ sont indépendantes et $\rho \in (0,1)$. 

\noindent Un défaut avant la date $t$ est défini par
\begin{equation}
\tau_i \le t
\quad \Longleftrightarrow \quad
X_i \le \Phi^{-1}\big(p(t)\big),
\end{equation}
où $p(t)$ est la probabilité de défaut marginale.

\noindent Conditionnellement à $Z=z$, on a
\begin{equation}
X_i \mid Z=z \sim \mathcal N(\sqrt{\rho}\,z,\,1-\rho),
\end{equation}
d’où
\begin{equation}
q(z,t)
:= \mathbb P(\tau_i \le t \mid Z=z)
=
\Phi\!\left(
\frac{\Phi^{-1}(p(t)) - \sqrt{\rho}\,z}{\sqrt{1-\rho}}
\right).
\end{equation}
Or nous savons que pour un notionnel unitaire et un taux de recouvrement constant $R$ la perte du portefeuille est
\begin{equation}
L_N(t)
=
(1-R)\,\frac{1}{N}
\sum_{i=1}^N \mathbf 1_{\{\tau_i \le t\}}.
\end{equation}
Et conditionnellement à $Z=z$, les variables
$\mathbf 1_{\{\tau_i \le t\}}$ sont i.i.d.\ de loi
$\mathrm{Bernoulli}(q(z,t))$.
Ainsi par la loi forte des grands nombres,
\begin{equation}
L_N(t)
\xrightarrow[N\to\infty]{\text{p.s.}}
L(t)
:=
(1-R)\,q(Z,t).
\end{equation}
Nous avons ainsi trouvé une formule explicite pour la perte du portefolio
entier en fonction de la corrélation implicite $\rho$. Pour calculer l'espérance sur $Z$
totale ou bien sur une tranche nous pourrons par exemple utiliser une méthode de
quadrature de Gauss-Hermite.

\begin{figure}[H]
\centering
\includegraphics[width=0.8\textwidth]{images/LHP_corr.png}
\caption{Perte du portefeuille en fonction de la corrélation implicite $\rho$ pour différentes tranches du CDX NA IG Series 19}
\label{fig:vasicek}
\end{figure}







