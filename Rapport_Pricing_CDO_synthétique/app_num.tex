\section{Application numérique}

\subsection{Tarification de tranches}

\subsubsection{Jeu de données}\label{subsubsec:data}

Pour notre étude nous étudierons $Q = 4$ tranches du CDO synthétique
CDX NA IG Series 19, il possède quatre différentes maturités (3, 5, 7
and 10 ans) et est basé sur un portefeuille de $n = 125$ contrats CDS.
Nous utiliserons les données fournies Table (\ref{tab:OkhrinXu}) par Okhrin et Xu (2017) pour la
maturité $T = 5$ ans sur 10 dates fournies entre 01/06/2014 et le 
15/03/2015.

\begin{table}[H]
\centering
\begin{tabular}{lccccc}
\toprule
Date & 0--3\% & 3--7\% & 7--15\% & 15--100\% & CDS \\
\midrule
2014/06/01 & 4.250 & 2.000 & 0.036 & 0.014 & 39 \\
2014/07/03 & 3.750 & 1.375 & 0.048 & 0.015 & 37 \\
2014/08/15 & 4.094 & 1.719 & 0.050 & 0.014 & 38 \\
2014/09/23 & 3.750 & 1.375 & 0.056 & 0.012 & 37 \\
2014/10/11 & 5.775 & 1.810 & 0.050 & 0.012 & 41 \\
2014/11/17 & 4.188 & 0.985 & 0.057 & 0.015 & 35 \\
2014/12/01 & 3.183 & 0.747 & 0.060 & 0.016 & 32 \\
2015/01/07 & 7.065 & 0.875 & 0.055 & 0.013 & 39 \\
2015/02/10 & 7.559 & 0.563 & 0.055 & 0.014 & 37 \\
2015/03/15 & 6.874 & 0.073 & 0.064 & 0.015 & 34 \\
\bottomrule
\end{tabular}
\caption{Spreads de 4 tranches du CDX NA IG Series 19 et les spreads des CDS associés}
\label{tab:OkhrinXu}
\end{table}

Le taux d'intérêt est pris constant à $r = 0.0014$ en accord avec la moyenne du LIBOR sur les
dates considérées et le taux de recouvrement est fixé à $R = 0.4$ en accord avec
la politique de l'entreprise Markit qui administre l'indice de ces produits. \cite{OhkrinXu}
Les taux de recouvrement moyens observés pour les obligations corporate senior unsecured sur 
longue période sont autour de 35--40 \%, selon les données historiques de Moody's Investors Service. Ces statistiques justifient l'usage conventionnel
d'un taux de recouvrement de 40 \% dans la calibration des modèles de CDS. \cite{Tresor}

\subsubsection{Copule gaussienne}

Dans cette partie nous allons exposer les premiers résultats obtenus pour un modèle simple de
copule gaussienne sous hypothèse d'un portefeuille homogène.

Les résultats de calibrations sur les données présentées en section (\ref{subsubsec:data}) sont
présentés Figure (\ref{fig:calibration_copule_gaussienne}) et montrent un résultat connu de la
littérature : le modèle gaussien présente des queues de distributions indépendantes et n'arrive
donc pas à simuler correctement les évènements rares de défauts corrélés, les tranches equity sont bien approximées
mais les autres tranches sont surévaluées par rapport aux spreads observés.

\begin{figure}[H]
\centering
\includegraphics[width=1\textwidth]{images/spread_gaussian_1fact.png}
\caption{Comparaison pour chaque tranche des spreads observés et calculés par copule gaussienne à un facteur}
\label{fig:calibration_copule_gaussienne}
\end{figure}

Nous pouvons voir Figure (\ref{fig:corr_copule_gaussienne}) que la corrélation implicite du modèle gaussien
est très élevée oscillant autour de 0,7 pour les différentes dates. Cette forte corrélation explique la 
surévaluation des tranches mezzanine et senior, le modèle gaussien doit compenser l'absence de queues de distribution
 dépendantes par une forte corrélation pour simuler les défauts corrélés nécessaires à la tarification de ces tranches.

\begin{figure}[H]
\centering
\includegraphics[width=0.8\textwidth]{images/spread_ev_gaussian_1fact.png}
\caption{Evolution du coefficient de corrélation implicite pour la copule gaussienne à un facteur}
\label{fig:corr_copule_gaussienne}
\end{figure}

La Figure (\ref{fig:smile_gaussian_1fact}) montre que la corrélation implicite par tranche forme
un "\emph{smile}" caractéristique des modèles à copule gaussienne, la tranche equity (0--3 \%) nécessite
une corrélation implicite importante, la tranche senior également, tandis que les tranches mezzanine 
(3--7 \% et 7--15 \%) nécessitent une corrélation implicite plus faible. Ce phénomène peut être expliqué
de deux manières différentes : soit le modèle de copule implicite est bien plus complexe que le modèle gaussien
ou bien le marché des tranches de CDO synthétiques est illiquide et les spreads observés ne sont pas fiables,
ce qui expliquerait les difficultés de calibration du modèle gaussien.

\begin{figure}[H]
\centering
\includegraphics[width=0.8\textwidth]{images/smile_gaussian_1fact.png}
\caption{Correlation implicite par tranche pour la copule gaussienne à un facteur}
\label{fig:smile_gaussian_1fact}
\end{figure}

\subsubsection{Copule de Student}

Nous présentons ici les résultats obtenus avec une copule de Student à un facteur, calibrée sur les
données décrites en section (\ref{subsubsec:data}). Par construction, la copule de Student introduit une
dépendance de queue (via le paramètre \(\nu\)), ce qui vise à mieux représenter les épisodes de stress où
des défauts peuvent survenir de manière plus simultanée que dans le cadre gaussien.

Les spreads de tranches obtenus après calibration sont présentés Figure (\ref{fig:calibration_copule_student}).
On observe que la tranche equity (0--3\%) est correctement reproduite par le modèle, ce qui est cohérent
avec le fait que cette tranche est principalement sensible aux premiers défauts. En revanche, le modèle
reste trop éloigné des spreads observés sur les tranches mezzanine et senior : la tranche 3--7\% est
surévaluée sur l’ensemble de la période, et l’écart est encore plus marqué pour la tranche 7--15\% dont
les spreads de marché sont très faibles sur la période considérée, alors que le modèle reste à des niveaux
élevés. La tranche 15--100\% est également surestimée, même si l’écart tend à se réduire en fin d’échantillon.
Ces résultats suggèrent que, malgré l’ajout de dépendance de queue, l’hypothèse de portefeuille homogène à
un facteur demeure trop restrictive pour ajuster simultanément l’ensemble des tranches.

\begin{figure}[H]
\centering
\includegraphics[width=1\textwidth]{images/Spread_tranches Student.png}
\caption{Comparaison pour chaque tranche des spreads observés et calculés par copule de Student à un facteur}
\label{fig:calibration_copule_student}
\end{figure}

La Figure (\ref{fig:corr_copule_student}) montre l’évolution de la corrélation implicite calibrée pour la
copule de Student. On constate que \(\rho\) prend des valeurs élevées sur une grande partie des dates
(autour de \(0.7\) en début de période), avant de diminuer nettement sur les derniers points (autour de \(0.4\)).
Comme dans le cas gaussien, cette corrélation élevée reflète le fait que, dans une calibration globale
à un paramètre, le modèle mobilise une dépendance forte pour générer suffisamment de pertes communes afin
d’expliquer la structure des spreads de tranches. Le passage à une copule de Student ne supprime donc pas
la nécessité d’une corrélation implicite élevée lorsque la calibration reste contrainte par une structure
homogène.

\begin{figure}[H]
\centering
\includegraphics[width=0.8\textwidth]{images/Correlation Student.png}
\caption{Evolution du coefficient de corrélation implicite pour la copule de Student à un facteur}
\label{fig:corr_copule_student}
\end{figure}

La Figure (\ref{fig:smile_student_1fact}) met en évidence un smile de corrélation implicite par tranche,
analogue à celui observé dans le cadre gaussien. La corrélation implicite est élevée pour la tranche
equity (attachement \(0\%\)) et atteint des valeurs très fortes au voisinage de l’attachement \(3\%\),
puis chute fortement pour l’attachement \(7\%\) avant de remonter pour l’attachement \(15\%\).
Ce profil indique que la dépendance requise par le marché n’est pas capturable par une unique corrélation
\(\rho\) : chaque tranche « demande » un niveau différent de dépendance afin de reproduire son spread, ce
qui se traduit par un smile persistant.

\begin{figure}[H]
\centering
\includegraphics[width=0.8\textwidth]{images/Smile Correlation_ Student.png}
\caption{Smile de corrélation implicite par tranche pour la copule de Student à un facteur}
\label{fig:smile_student_1fact}
\end{figure}

Enfin, la Figure (\ref{fig:smile_moyen_student}) compare le smile moyen (moyenne des corrélations implicites
par attachement) à la corrélation globale moyenne issue de la calibration. On observe que la corrélation
globale moyenne ne reflète pas correctement la structure en smile, en particulier la forte exigence de
dépendance sur les tranches basses et la remontée sur les tranches plus seniors. Cela confirme que, même
avec une copule de Student, une spécification plus flexible est nécessaire pour améliorer l’ajustement des
tranches mezzanine et senior (par exemple une structure sectorielle, une copule hiérarchique, ou des
mélanges de copules).

\begin{figure}[H]
\centering
\includegraphics[width=0.8\textwidth]{images/SMILE moyen_ Student.png}
\caption{Smile moyen et corrélation globale moyenne pour la copule de Student à un facteur}
\label{fig:smile_moyen_student}
\end{figure}
